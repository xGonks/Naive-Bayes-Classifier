
\documentclass[twocolumn]{article}

\usepackage{graphicx} % Required for inserting images
\usepackage{amsmath}
\usepackage{amssymb}
\usepackage{url}
\usepackage[a4paper, left=1.5cm, right=1.5cm, top=2.5cm, bottom=2.5cm]{geometry}
\usepackage{etoolbox}
\usepackage{booktabs}

\patchcmd{\thebibliography}{\section*{\refname}}{}{}{}

\title{Naive Bayes classifier applied to paranormal stories}

\author{
Alan Fernando Bravo Pimentel \and
Gonzalo Makenly Higuera Inzunza \and
Juan Pablo de la Peña Gonzalez \and
Sarah Camila Guzmán Fierro \and
Miguel Angel Rivas Torres
}

\date{Semptember 14th 2025}

\begin{document}

\maketitle

\begin{abstract}
Online websites act as repertoires of reports of paranormal experiences, offering a distinctive set of information on cultural beliefs and individual perceptions. This study aims to  explore the use of Natural Language Processing (NLP) techniques to categorize such stories. A Naive Bayes classifier was trained to predict the type of phenomena described. The data acquisition process involved web scraping and drew information from the \textit{Your Ghost Stories} platform, the text was then processed using a count vectorizer, and a sparse matrix for showing frequencies of words was produced. Several iterations of the Naïve Bayes classifier were tested, and several accuracy metrics were utilized for evaluating the model's performance.  

The findings suggest that \_\_\_\_\_\_\_\_\_\_\_\_\_\_\_\_\_\_\_\_\_. More precisely, it was noticed that \_\_\_\_\_\_\_\_\_\_\_\_\_\_\_\_\_\_\_\_\_.  


\end{abstract}

\section{Introduction}
Throughout human history, paranormal narratives have captured the imagination of individuals all over the world. Stories of supernatural phenomena, haunted places, and spiritual encounters have been commonplace elements of folklore and constitute a form of cultural data. For this reason, several websites and platforms have emerged with the mission of storing and showcasing this wide variety in the digital era. Among these, the platform \textit{Your Ghost Stories} offers thousands of individual accounts of paranormal experiences submitted by users. Undeniably, these stories provide a highly interesting and rich framework for applying computational tools to explore, analyze, and structure information in the form of text.
The primary goal of this study is to train and implement a naive Bayes classifier capable of categorizing paranormal stories according to the type of phenomenon described. In order to accomplish this, we built a dataset by performing web scraping on the website \textit{Your Ghost Stories}. Afterward, some techniques of natural language processing were employed, such as tokenization, and a sparse matrix prepared the data for further analysis. 
A naive Bayes classifier was then appiled to this dataset, taking advantage of its effectiveness for text classification applications. By modeling the distribution of words from different categories of these stories, the classifier can predict the type of event described in each story. The performance of this classifier was later evaluated through several metrics. \\
This study demonstrates the capabilities of NLP techniques coupled with probabilistic modeling in text to explore the way that paranormal stories are shared and categorized.


\section{Methodology}
In order to acquire the data for the classifier, we employed web scraping to extract narratives from the \textit{Your Ghost Stories} platform. The first step was to verify scraping permission using the paths\_allowed() function from the robotstxt library in R. Then, the script used identified the relevant elements to capture such as the title of each story, its description as well as metadata such as country, state, and category. Additionally, a custom function called get\_story() was implemented so as to automate the process and extract multiple stories in parallel, which increased efficiency. The resulting dataset was structured into an organized table containing fields for id, title, country, state, category and description. Finally, the data were exported to a CSV file. \\
The preparation of the text for analysis began with each story being tokenized in turn, allowing the narrative descriptions to be broken down into discrete words. Then, the most common stop words were eliminated and Only the most instructive terms remained in the text. A frequency count of the words that appeared in each document was then produced, offering a systematic method of  expressing the narratives numerically. These word counts were then transformed into a sparse matrix, where each row corresponded to a story and each column represented a unique word (feature). The entries of the matrix indicated the frequency of each word inside a particular story. 
Finally, the categories associated with each story were aligned with the rows of the newly created sparse matrix, so that we could create a dataset in which we could represent the information quantitatively and keep it linked to its respective class label.
Afterward, a naive Bayes Classifier was trained on the dataset. This classifier relies the assumption of conditional independence between features, which in practice means that the presence of one word in a document is considered independent of the presence of any other word given the category. The classifier also relies on the Bayes Theorem to calculate the probability that a certain observation belongs to a class.
In particular, given a new observation, 
\( \tilde{\boldsymbol{x}} = (\tilde{x}_{1}, \dots, \tilde{x}_{p})^{\intercal}\tilde{\boldsymbol{x}}.\)
we compute
\[
\mathbb{P}(Y = k \mid \boldsymbol{X} = \tilde{\boldsymbol{x}}) = \frac{\mathbb{P}(Y = k)f(\tilde{\boldsymbol{x}} \mid y = k)}{\sum_{k = 1}^{K} \mathbb{P}(Y = k)f(\tilde{\boldsymbol{x}} \mid y = k)}
\]



The implementation was carried out using the "naivebayes" package in R. As per usual, the training set was defined by randomly sampling 70\% of the data, while the other 30\% was reserved for testing. The performance of our classifier was then evaluated using accuracy, precision, recall, and F1-score to capture overall performance when predicting and performance in individual categories.
Finally, certain improvements were made to fix common problems with text classification.  The Poisson distribution was investigated so that it could be used as a substitute for the standard assumption in the naive Bayes model as word counts are discrete and sometimes sparse. Another tool employed was Laplace smoothing and it was also used to manage the zeros in the matrix and prevent zero classification probabilities. Cross-validation techniques were also used and helped us find the smoothing parameter's most optimal value and improve the performance of the classifier.



\section{Application}
During the web scaping process, up to 28,600 different observations/stories were retrieved. 

\section{Conclusions}
The stories were 


\section{References}

\begin{thebibliography}{9}
\setlength{\itemsep}{0pt}
\setlength{\parskip}{0pt}


\bibitem{RPubs}
RPubs - Naïve Bayes con R para clasificacion de texto. (n.d.).
\url{https://rpubs.com/jboscomendoza/naive_nayes_con_r_clasificacion_texto}

\vspace{0.3cm}


\end{thebibliography}

\end{document}
